\chapter*{Bits and Algebra}
\begin{definition}
    [Sign Extension]
    When you extend the sign bit to the left to fill in the rest of the bits.
\end{definition}

\begin{example}
    \begin{align*}
          & 0000\,0011 (8b)                  \\
        + & \textcolor{red}{1111}\,1111 (4b) \\
        \hline                               \\
    \end{align*}
    In this case, the leftmost bit, the sign bit, of $1111$ is extend
\end{example}

\begin{definition}
    [Hexadecimal]
    A base-16 number system. \\
    $0, 1, 2, 3, 4, 5, 6, 7, 8, 9, A, B, C, D, E, F$
\end{definition}

\section{Floats in C}
\begin{itemize}
    \item \texttt{float} is 32 bits
    \item \texttt{double} is 64 bits
    \item \texttt{FP16/BFloat16} is 16 bits (Not default in C)
\end{itemize}

\begin{definition}
    [Floating Point Standard]
    A certain number of bits are used to represent the mantissa (fraction) and the exponent (distance from least significant bit) of a number. \\
\end{definition}

\begin{example}
    [Converting Binary Integers to Float]
    \begin{align*}
        1110\,0001\,.1000 &                                                \\
        \text{Sign}       & = 1                                            \\
        \text{Mantissa}   & = 1110                                         \\
        1110\,0001\,.1000 & \approx 1110\times 2^4 \to \text{Exponent} = 4
    \end{align*}
    \begin{align*}
        0000.\,0001\,1101\,1001                                                      \\
        \text{Sign} = 0                                                              \\
        \text{Mantissa} = 1110                                                       \\
        0000.\,0001\,1101\,1001 & \approx 1110\times 2^{-7} \to \text{Exponent} = -7 \\
    \end{align*}

\end{example}

\begin{definition}
    [IEEE 754]
    A standard for floating point numbers. \\
    \begin{itemize}
        \item 1 bit for sign
        \item 8 bits for exponent
        \item 23 bits for mantissa
    \end{itemize}
\end{definition}

\begin{theorem}
    [IEEE 754]
    \begin{align*}
        \text{Value} & = (-1)^{\text{Sign}} \times 1.\text{Mantissa} \times 2^{\text{Exponent} - 128}
    \end{align*}
\end{theorem}

\begin{example}
    [IEEE 754 Example]
    \begin{align*}
        0\,0111\,1111\,1100\,0000\,0000\,0000\,0000 \\
    \end{align*}
\end{example}